\documentclass{beamer}

\usepackage{comment}

\usepackage{caption}

% Essential packages
\usepackage{capt-of}
\usepackage{setspace}
\usepackage{bbm}
\usepackage{caption}
\setbeamertemplate{caption}[numbered]
\usepackage{hyperref} % Load hyperref last

\usepackage{amsfonts}

% Color definitions - explicitly define BrickRed
\definecolor{MainColor}{RGB}{178, 34, 52} % Standard BrickRed RGB values
\definecolor{AlertColor}{RGB}{178, 34, 52} %{0, 0, 128}  NavyBlue or BrickRed
\definecolor{ExampleColor}{RGB}{111, 33, 205}

\usepackage{hyperref}
\usepackage{booktabs}
\usepackage{tikz}
\usetikzlibrary{positioning, arrows.meta}

\usepackage[most]{tcolorbox}

\usepackage[backend=bibtex, style=authoryear]{biblatex}
\newcommand{\coloredciteyear}[1]{\textcolor{MainColor}{\citeyear{#1}}}
\bibliography{bib_riha}

\newcommand{\citenopara}[1]{\citeauthor{#1}~\citeyear{#1}}

\AtBeginSection[]{ \begin{frame}\centering \LARGE \insertsection\end{frame} }

% Hyperlink setup
\hypersetup{
	colorlinks=true,
	linkcolor=MainColor,
	urlcolor=MainColor,
	citecolor=MainColor
}

% Theme settings
\usetheme{default}
\usecolortheme{default}

% Custom color settings
\setbeamercolor{palette primary}{fg=MainColor}
\setbeamercolor{palette secondary}{fg=MainColor}
\setbeamercolor{structure}{fg=MainColor}
\setbeamercolor{title}{fg=MainColor}
\setbeamercolor{frametitle}{fg=MainColor}
\setbeamercolor{section in toc}{fg=MainColor}
\setbeamercolor{subsection in toc}{fg=MainColor}
\setbeamercolor{itemize item}{fg=MainColor}
\setbeamercolor{itemize subitem}{fg=MainColor}
\setbeamercolor{enumerate item}{fg=MainColor}
\setbeamercolor{enumerate subitem}{fg=MainColor}
\setbeamercolor{button}{bg=MainColor}
\setbeamercolor{alerted text}{fg=AlertColor}
\setbeamercolor{example text}{fg=ExampleColor}

% Itemize bullet customization
\setbeamertemplate{itemize item}[circle]
\setbeamertemplate{itemize subitem}{\textasteriskcentered}

% Bold text in main color
\renewcommand{\bfseries}{\color{MainColor}
	\fontseries{b}\selectfont}
\newcommand{\textbfv}[1]{\textbf{\textcolor{ExampleColor}{#1}}}

% Presentation information
\title{Computational Economics}

\subtitle{ \vspace{0.5em} Lecture 1: An Intro to Dynamic Stochastic General Equilibrium Models}
	  
\author[\small Briand,]{
	\small
	Etienne Briand\\
}

\institute[]{Concordia University}
\date{Winter 2026}


 %First, economics uses formal mathematical models that are idealized versions of reality to study human behavior. One of my aims when teaching is for students to understand why formal models, because they are simplified versions of reality can be useful tools for understanding complex real-world phenomena. Second, since there is no single model that can be applied to all settings, I want to students to gain the ability to judge whether a given model is appropriate for addressing a specific question.
 
 %The Lucas critique argues that it is naive to try to predict the effects of a change in economic policy entirely on the basis of relationships observed in historical data, especially highly aggregated historical data.[1] More formally, it states that the decision rules of Keynesian models—such as the consumption function—cannot be considered as structural in the sense of being invariant with respect to changes in government policy variables.[2] It was named after American economist Robert Lucas's work on macroeconomic policymaking.
% The Lucas critique is significant in the history of economic thought as a representative of the paradigm shift that occurred in macroeconomic theory in the 1970s towards attempts at establishing micro-foundations.

\begin{document}
	\begin{frame}
		\titlepage
	\end{frame}

   \begin{frame}
   	\frametitle{Motivation}
   	
   	The \alert{Lucas critique} (1976) warns that policy evaluation cannot rely on historical empirical relationships. This argument changed how to (not) conduct macroeconomic research and led to the emergence of \alert{DSGE} models.
   	\medskip
   	
   	DSGEs provide a disciplined (and simplified) framework to study real-world phenomena.
   	\medskip
   	 
   	 Importantly, these models are \alert{microfounded}: aggregate dynamics are consistent with individuals optimizing given constraints and expectations.
   	
   \end{frame}

     \begin{frame}
   	\frametitle{Motivation}
   	
   	DSGE models typically cannot be solved using pen and paper. This class will cover methods to tackle this issue, allowing us to use these models to study:
    \bigskip

   	\begin{itemize}
   		\item Dynamic responses to shocks.
   		\medskip
   		
   		\item Propagation mechanisms.
   		\medskip
   		
   		\item Counterfactuals and policy experiments.
   		\medskip
   		
   		\item Historical decomposition.
   	\end{itemize}
   	
   \end{frame}
   
	
    %todo: section with the name of the model 
    
    \section{A Simple Model}
    
    \begin{frame}
    	\frametitle{Overview}
    	
    	As an intro to D(S)GE models, we study an optimal consumption–saving problem.
    	The model is effectively a deterministic RBC framework in which labor input is set exogenously to one.
    	\bigskip 
    	
    	This environment remains the backbone of state-of-the-art macroeconomic models.
    	\vfill
    	
    	\footnotesize
    	\alert{Note}: In the absence of uncertainty, one could drop the ``S” from DSGE.
    	
    \end{frame}
    
	\begin{frame}
		\frametitle{Economic Environment}
		
		\begin{itemize}
			
			\item Representative household period utility: $ U(C_t) = \tfrac{C_t^{1-\gamma}-1}{1-\gamma}$
			\medskip
			
			\item Discount factor: $0< \beta <1$.
			\medskip
			
			\item Aggregate production function:  $	Y_t = K_{t-1}^\alpha$
			\medskip
		
			\item Law of motion for capital: $K_t = (1-\delta)K_{t-1} + I_t$
			\medskip
			
			\item Aggregate resource constraint: $Y_t = C_t + I_t$
			
		\end{itemize}
		
	\end{frame}

    \section{Social Planner's Problem}

	\begin{frame}
	\frametitle{Social Planner's Problem}
	
   We solve the consumption–saving problem from the social planner’s perspective. In a frictionless environment like the one above, the solution coincides with the decentralized equilibrium, where households and firms maximize taking prices as given.
	\bigskip
	
	The social planner maximizes the discounted sum of period utility, taking as given the aggregate production function, aggregate resource constraint and the capital's law of motion.


   \end{frame}

	\begin{frame}
	%todo: dont call it the sequential approach ... 
	\frametitle{Social Planner's Problem (cont'd)}  %Sequential Approach
	
	Formally the social planner's problem solves:
	
	\begin{equation}
		\max_{\{C_t,K_t,I_t\}_{t=0}^\infty}    \sum_{t=0}^\infty  \beta^t  \alert{ E_t} U(C_t)
	\end{equation}
	subject to
	
	\begin{equation}
	    	Y_t = K_{t-1}^\alpha
	\end{equation}

    \begin{equation}
	     K_t = (1-\delta)K_{t-1} + I_t
    \end{equation}
	
	\begin{equation}
	     Y_t = C_t + I_t
	\end{equation}

    \begin{equation}
    	K_{-1} \; \text{given}.
    \end{equation}

	\vfill
  \footnotesize
   \alert{Note}: As mentionned earlier, there is no source of uncertainty in this model, and thus the use of the expectation operator, $E_t(\cdot)$ is not required. %formally 

	
    \end{frame}


   	\begin{frame}
   \frametitle{Social Planner's Problem (cont'd)}%Sequential Approach (cont'd)
    
    Combining the constraints, we get: 
    
    \begin{equation}
    	\max_{\{C_t,K_t\}_{t=0}^\infty}   \sum_{t=0}^\infty  \beta^t    \tfrac{C_t^{1-\gamma}-1}{1-\gamma}
    \end{equation} %E_t
    subject to
    
    \begin{equation}
       K_{t-1}^\alpha = C_t  + K_t - (1-\delta)K_{t-1} 
    \end{equation}
    
     \begin{equation}
    	K_{-1} \; \text{given}.
    \end{equation}
    \medskip 
    
   
   \vfill
    
    \footnotesize
    \alert{Note}: $C_t = K_{t-1}^\alpha - K_t + (1-\delta)K_{t-1}$, thus it would be possible to replace in the objective and solve for an unconstrained problem.
    
    \end{frame}


\begin{frame}
	\frametitle{Social Planner's Problem (cont'd)}
	
	We can write this problem as a Lagrangian:
	
	\begin{equation}
		\mathcal{L} =  \sum_{t=0}^\infty \beta^t  \left\{ \tfrac{C_t^{1-\gamma}-1}{1-\gamma} + \lambda_t \left[ K_{t-1}^\alpha - C_t  - K_t +(1-\delta)K_{t-1}  \right] \right\}.
	\end{equation} % E_t 
	
   \end{frame}

    \begin{frame}
    \frametitle{Optimality Conditions}
    
    Taking the derivatives of $\mathcal{L}$ wrt to the controls, $C_t$ and $K_t$, we get the FOCs:
    
    \begin{equation} 
        C_t^{-\gamma} = \lambda_t
    \end{equation} 

    \begin{equation} 
    	\lambda_t = \beta \lambda_{t+1}  \left[ \alpha K_{t}^{\alpha-1} + (1-\delta)\right].
    \end{equation} 
     \smallskip
    
    %TODO: check these...
    The KKT are given by:
     \begin{equation} 
    	\lambda_t \left[ K_{t-1}^\alpha - C_t  - K_t +(1-\delta)K_{t-1} \right] = 0
    \end{equation} 

    \begin{equation} 
	\lambda_t \geq 0, \left[ K_{t-1}^\alpha - C_t  - K_t +(1-\delta)K_{t-1}\right] \geq 0.
    \end{equation} 

 
    \end{frame}


 \begin{frame}
	\frametitle{Optimality Conditions (cont'd)}
	
	%By optimality, 
	The aggregate resource constraint always holds at equality, thus $\lambda_t >0 \; \forall t$ and with $K_{-1}>0$, we never encouter corner solutions. 
	%and C_t > 0 \forall t by the first FOC
	\bigskip
	
	Optimality implies: 
	\bigskip
	
	$C_t^{-\gamma} = \beta C_{t+1}^{-\gamma}[\alpha K_{t}^{\alpha-1} + (1-\delta)]$ \hfill [Euler Eq.]
	\bigskip
    
    with 
    \bigskip
    
    $K_{t-1}^\alpha = C_t  + K_t -(1-\delta)K_{t-1}$  \hfill [Aggr. resource constr.]
    
  
	%todo: TVC??
	
\end{frame}


\section{Equilibrium}


    \begin{frame}
   	\frametitle{Definition}
   	 
   	 Formally, the definition of an equilibrium consists of a list of \alert{objects} and \alert{conditions} that these objects must satisfy.
   	 \bigskip
   	 
   	 \alert{Definition}: Given an initial stock of capital $K_{-1}$, a sequential equilibirum to the consumption-saving problem is:
   	 \medskip  
   	 
   	 \begin{itemize}
   	 	\item An allocation $\Omega:=\{C_t,K_t\}_{t=0}^\infty$ for the social planner.
   	 \end{itemize}
   	 	 \medskip  
   	 	 
   	 such that:
    	 	 \medskip  
    	 	 
   	 \begin{itemize}
   	 \item[1.]	 The  allocation $\Omega$ maximizes the discounted of sum of period utilities.
   	 	 \medskip  
   	 	 
   	 \item[2.] The aggregate resource constraint holds at equality
   	 	 \begin{enumerate}
   	   \item[] $K_{t-1}^\alpha = C_t  + K_t -(1-\delta)K_{t-1}$.
   	   	 \end{enumerate}
   	 
   	  \end{itemize}
   	
   \end{frame}

    \begin{frame}
   	\frametitle{Characterization}
   	
   	The characterization of an equilibrium consists of listing all of the mathematical conditions necessary compute the social planner's allocation $\Omega$.
   	\bigskip
   	
   	In the present case, the Euler equation and the aggregate resource constraint are sufficient.
   	
   \end{frame}
   

\section{Steady-state}

%TODO: i want a section for the nsss but im not sure how to introduce it w/o talking about the dynamics first.
\begin{frame}
	\frametitle{The Steady-State} %(non-stochastic)
		
	 In a steady-state, all endogenous variables are constant over time and satisfy the model’s equilibrium conditions. 
	 \bigskip
	 
	 We can compute the values for these variables by dropping the time index in the FOC and constraints:
	
	\begin{equation} 
		%K = \left(\frac{\delta}{\alpha}\right)^{\tfrac{1}{\alpha-1}}
		%	\delta =   \alpha K^{\alpha-1} 
		K = \left( \frac{\beta^{-1}-(1-\delta)}{\alpha}\right)^{\tfrac{1}{\alpha-1}}
	\end{equation} 
 
   \begin{equation} 
   	I = \delta K 
   \end{equation} 

    \begin{equation} 
    	C= K^\alpha - I 
    \end{equation} 
    
     \begin{equation} 
    	\lambda = C^{-\gamma} 
    \end{equation} 
	
    \end{frame}

\section{Dynamics}

\begin{frame}
	\frametitle{Solving for the Transition Dynamics}
	
	Our goal is solve for the dynamics associated with the pair of non-linear first-order difference equations formed by the Euler eq. and the resource constraint (obviously for a given $K_{-1} \neq K$).
	\bigskip
	
	%todo; re-edit
	We can show that this economy has exactly \alert{one stable manifold}, meaning that for a given $K_{-1}$, consumption and capital converge to their steady-state values only if $C_t$ is chosen optimally $\forall t$.


\end{frame}
   
   
   \begin{frame}
   	\frametitle{Shooting Algorithm}
   	
   
   In other words, given $K_{-1}$, there exists a unique path
   $\{C_t^*,K_t^*\}_{t=0}^T$ that satisfies optimality and converges to the steady state
   $\{C,K\}$.
   \medskip
   
   The challenge is that $C_0^*$ is not pinned down by the eq. cond.
   \medskip
   	
   	We can solve for this path using a \alert{shooting algorithm}:
   \medskip 
   
   \footnotesize
   \begin{itemize}
   	
   	\item[1.] Guess a value for $C_0 \in [0, K_{-1}^\alpha]$.
   	
   	\item[2.] Given $(C_0, K_{-1})$, use the Euler equation and the aggregate resource constraint to compute $(C_{1}, K_0)$. Iterate forward up to $t=T$.
   	
   	\item[3.] If $K_T < K$, set the guess for $C_0$ as the upper bound of the interval; otherwise, set it as the lower bound.
   	
   	\item[4.] Repeat steps 1–3 until convergence \\(i.e., $K_T=K$ and $C_T=C$).
   \end{itemize}
   	\medskip 
   	
   	\normalsize
   This procedure transforms a two–point boundary value problem into an initial–condition problem.
   	
   	%we have to do it for any K_-1
   	
   \end{frame}

   %TODO: results/notebooks


   \begin{frame}
  	\frametitle{Shooting Algorithm (cont'd)}
  	
  The shooting algorithm is not practical: (i) it solves for a \alert{single path} and (ii) it becomes computationally burdensome (infeasible) without \alert{perfect foresight}.
  % in the presence of uncertainty.
   %It delivers $C_0^*$ only for the given $K_{-1}$ and must be repeated for each initial capital level of interest. 
  % Moreover, the algorithm performs well under \alert{perfect foresight}, but quickly becomes computationally infeasible in the presence of uncertainty.
   \bigskip
   
   It is therefore natural for us to seek a solution in the form of a policy function that:
   \medskip 
   
   \begin{itemize}
   	\item  Maps the state $K_{t-1}$ to the control $C_t$ for all admissible values of the state.
   	\medskip 
   	
   	\item Accounts for uncertainty about future outcomes.
   \end{itemize}
   
    
  	
  \end{frame}
  
	
\end{document}
	
%TODO: add the other method (quasi newton)
	
	
		%In other words, given $K_{-1}$, our goal is to solve for a path $\{C_t^*,K_t^*\}_{t=0}^T$ that satisfies optimality and converge to $\{C,K\}$.
	%we have to do it for any K_-1
	
	
	
	
	

