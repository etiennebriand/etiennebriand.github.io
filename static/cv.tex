\documentclass{article}
\usepackage[utf8]{inputenc}
\usepackage{amsmath}
\usepackage{graphicx}
\usepackage{float}
\usepackage{booktabs}
\usepackage{amssymb}

%\usepackage{rotating}
%\usepackage{tikz}
%\usetikzlibrary{decorations}
%\usetikzlibrary {arrows.meta} 

\usepackage[dvipsnames]{xcolor}
%\definecolor{NavyBlue}{rgb}{0.0, 0.0, 0.5}


\usepackage{hyperref}
\hypersetup{
	colorlinks=true,
	urlcolor=NavyBlue%blue
}

\newcommand\setItemnumber[1]{\setcounter{enumi}{\numexpr#1-1\relax}}

\usepackage{caption} 
\captionsetup[table]{skip=10pt}

\title{\huge ETIENNE BRIAND}
\author{Economics Department, UQAM $\diamond$ Montreal, QC \\
              \href{mailto:etiennebriand1@gmail.com}{etiennebriand1@gmail.com} $\diamond$ 
          \href{https://etiennebriand.github.io/}{etiennebriand.github.io}}
\date{October 2025}


\begin{document}
	
	\maketitle
	
	%TODO: add research expertise. PRIMARY: macroeconoics, information economics. SECONDARY, macroeeconometrics. 
	%TODO: links to reasearch and teaching statements	
   	\noindent\textbf{\large EDUCATION} 
	\vspace{4pt}
	\hrule
	\vspace{6pt} 
	
	\noindent \textbf{Université du Québec à Montréal} \hfill 2020-2026(\textit{expected})\\
	\noindent Ph.D. in Economics
	\vspace{4pt}

	\noindent \textbf{Université de Montréal} \hfill 2018-2019\\
	\noindent M.Sc. in Economics
	\vspace{4pt}

	\noindent \textbf{Université de Montréal} \hfill 2013-2017\\
	\noindent B.Sc. in Economics and Mathematics
	\vspace{6pt} 
   
	\noindent\textbf{\large ACADEMIC VISITS} 
	\vspace{4pt}
	\hrule
	\vspace{6pt} 
	\noindent\textbf{Cornell University}  \hfill 2024-2025\\
	Faculty sponsor Mathieu Taschereau-Dumouchel
	\vspace{6pt}
	
	\noindent\textbf{\large RESEARCH} 
	\vspace{4pt}
	\hrule
	\vspace{6pt} 
	
	\noindent\underline{\textbf{Job Market Paper}}
	\vspace{6pt}
	
	%TODO: clickable links to papers
	\noindent\href{https://etiennebriand.github.io/riha.pdf}{\textbf{Rationally Inattentive Heterogeneous Agents}}
	%\noindent\textbf{Rationally Inattentive Heterogenous Agents}
	\vspace{6pt}
	
	\noindent \textbf{Abstract.} We solve business cycle models with rationally inattentive heterogeneous agents and compare their predictions with the data. [...] Models with standard labor market structures cannot simultaneously induce persistence in macro variables and cross-sectional expectations that match the data. This conundrum arises because no model is able to generate losses from intra- and intertemporal decisions of similar magnitude \textit{and} persistence in the growth rate of labor income. Moreover, conducting the same policy experiment in both models leads to starkly different conclusions. We discuss modifications to the models’ microfoundations, such as wages set by unions and market power on the side of firms, as possible ways to jointly match micro and macro evidence, eliminating the need to compromise between models that perform well in only one dimension.
	\vspace{.5em}
	
	\noindent\underline{\textbf{Working Papers}}
	\vspace{6pt}
	
	%\noindent\textbf{Inflation, Attention and Expectations}
	\noindent\href{https://etiennebriand.github.io/BMS_AttentionInflationExpectations.pdf}{\textbf{Inflation, Attention and Expectations}}
	\\ \textit{with Massimiliano Marcelino \& Dalibor Stevanovic}
	\vspace{6pt}
  
	\noindent\underline{\textbf{Work In Progress}}
	\vspace{6pt}
		
	\noindent \textbf{Shocks and their Propagation under Rational Inattention} 
	\vspace{4pt} 
	
	\noindent \textbf{Abstract.} We estimate the parameters of a business cycle model with rational inattention to match the impulse responses to a monetary policy shock from an estimated medium-scale New Keynesian DSGE, holding the exogenous stochastic processes fixed. we compare the impulse responses of the two models to the remaining macroeconomic shocks. [...] alternative sources of inertia in prices and quantities lead to different conclusions about the main drivers of business cycles and the propagation of shocks.
	\vspace{4pt}
	
	\noindent \textbf{Quantifying the Effect of Noisy News on Business Cycles}\\
	\textit{with Patrick Fève \& Alain Guay}
	\vspace{4pt}
	
	\noindent \textbf{Abstract.}  We investigate the impact of noisy news shocks about aggregate TFP on business cycle dynamics. We begin by proposing a simple semi-parametric statistic that combines moment conditions between noisy signals and present or future changes in TFP to estimate the noise-to-signal ratio and the impulse response function of news and noise shocks. [...]
	\vspace{4pt}
	
	\noindent \textbf{Are Volatility Shocks Undertainty Shocks?}
	\vspace{4pt}
	
	\noindent \textbf{Abstract.} We study the impact of volatility shocks, identified by leveraging a combination of a proxy-VAR approach and DSGE-based instruments, on uncertainty and macroeconomic outcomes. 
   \vspace{6pt}
	
	\noindent\textbf{\large TEACHING EXPERIENCE} 
	\vspace{4pt}
	\hrule
	\vspace{6pt} 
	\noindent \textbf{Université du Québec à Montréal}
	
	\noindent Advanced macroeconomics I (graduate), TA, (2024)  %(ECO7011) graduate course,
	\vspace{4pt}
	
	\noindent Advanced macroeconomics II (graduate), TA, (2023)  %(ECO9011) graduate course,
	\vspace{4pt} 
	
	\noindent Methods of dynamic programming (graduate), TA, (2021) %(ECO9015),  graduate course,
	\vspace{4pt}
	
	\noindent Business cycles and economic policies (graduate), TA, (2021-2024)
	\vspace{4pt}
	% (ECO8061), graduate course,
	
	\noindent Macroeconometrics (graduate), TA, (2021-2024) %(ECO9036) graduate course, 
    \vspace{6pt}
    
	\noindent\textbf{\large PRESENTATIONS} (including scheduled)
	\vspace{4pt}
	\hrule
	\vspace{6pt} 
	\noindent\textbf{2025:} Bank of Canada Montreal Workshop $\cdot$ Barcelona School of Economics Summer Forum, Monetary Policy Workshop $\cdot$ 7th Behavioral Macroeconomics Workshop $\cdot$ Bank of Canada Graduate Student Paper Award
	\vspace{4pt}
	
	\noindent\textbf{2024:} Macro Lunch Cornell $\cdot$ 58th annual Canadian Economics Asociation Meetings $\cdot$ 63th annual congress of the Canadian Economic Society 
	\vspace{4pt}
	
    \noindent\textbf{2023:} 62th annual congress of the Canadian Economic Society  
   \vspace{6pt} 

	\noindent\textbf{\large SCHOLARSHIPS \& AWARDS} %REASEARCH GRANTS
	\vspace{4pt}
	\hrule
	\vspace{6pt} 
	\noindent\parbox[t]{0.65\textwidth}{Finalist Bank of Canada's Best Graduate Paper Award}%
	\hfill
	2025 (\textit{results pending})\\
	\vspace{4pt}
	
	\noindent\parbox[t]{0.65\textwidth}{Social Sciences and Humanities Research Council Doctoral Fellowships}%
	\hfill
	2022-2025\\
	\vspace{4pt}
	
	\noindent\parbox[t]{0.65\textwidth}{Excellence PhD scholarship, UQAM Econ Department}%
	\hfill
	2020-2022
	\vspace{6pt}
	\newline%\newline\newline\newline\newline\newline\newline\newline
	
	\noindent\textbf{\large REFERENCES} 
	\vspace{4pt}
	\hrule
	\vspace{6pt} 
	\noindent\parbox[t]{0.45\textwidth}{\textbf{Prof. Alain Guay (advisor)}\\
		Université du Québec à Montréal\\
		\href{mailto:guay.alain@uqam.ca}{guay.alain@uqam.ca}
	}
    \hfill 
    \noindent\parbox[t]{0.45\textwidth}{\textbf{Prof. Pavel Sevcik (advisor)}\\
    	Université du Québec à Montréal\\
    	\href{mailto:sevcik.pavel@uqam.ca}{sevcik.pavel@uqam.ca}
    }%
   \vspace{10pt}
   
   \noindent\parbox[t]{0.45\textwidth}{\textbf{Prof. Dalibor Stevanovic}\\
   	Université du Québec à Montréal\\
   	\href{mailto:dstevanovic.econ@gmail.com}{dstevanovic.econ@gmail.com}
   }%


	\vspace{6pt}
	
	\noindent\textbf{\large PERSONAL} 
	\vspace{4pt}
	\hrule
	\vspace{6pt} 
	\noindent Languages: French, English
	\vspace{4pt}
	
	\noindent Citizenship: Canadian
	\vspace{4pt}
	
    \noindent \textbf{Member of the Canadian National Judo Team} \hfill 2012-2021\\
    \noindent 9th place World Championship Tokyo.
	
	
\end{document}


\noindent \textbf{Revisiting Business Cycles under Rational Inattention} 
\vspace{.5em} 

\noindent \textbf{Abstract.} We solve a dynamic stochastic general equilibrium model with rational inattention (an RI-DSGE). The physical environment resembles a simple New Keynesian model with capital. In the model, shocks to the marginal efficiency of investment induce business cycle comovement due to limited attention. By contrast, several bells and whistles are needed to achieve the same result in conventional DSGE [...]
\vspace{.5em}

	\noindent\textbf{\large PERSONAL INFORMATION} 
\vspace{4pt}
\hrule
\vspace{6pt}
\noindent \textbf{Citizenship} \hfill Canadian\\
\vspace{4pt} 
\noindent \textbf{Languages} \hfill French, English\\
\vspace{4pt} 
